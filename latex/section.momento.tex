\section{Momentos}

Momentos são medições escalares usadas para caracterizar uma função e capturar suas características mais significativas. São bastante usados a centenas de anos em estatística para descrever a forma de uma função de densidade probabilística e em corpos rígidos para medir a distribuição de massa.
Do ponto de vista matemático, momentos são "projeções" de uma função em uma base polinomial (da mesma forma que, transformada de Fourier é uma projeção em uma base de funções harmônicas).

Definindo então uma imagem como sendo uma função real \( f(x,y) \) de duas variáveis em a compact support \( D \subset \mathbb{R} \subset \mathbb{R} \) e tento uma integral finita diferente de zero. Podemos definir o momento de forma genérica \( M_{pq}^{(f)} \) de uma imagem \( f(x, y) \), onde \( p, q \) são valores não negativos e inteiros e \( r = p + q \) é a ordem do momento, como:

\[ M_{pq}^{(f)} = \iint_D p_{pq} (x, y) f(x, y) \,dx \,dy \]

\section{Momentos invariantes em translação, rotação e escala}

\subsection{Introdução}

Translação, rotação e escala (abreviado como TRS, do inglês \textit{Translation, rotation and scaling}) são as transformações de coordenadas espacial mais simples. TRS é uma transformada de 4 parâmetros, que pode ser descrita como

\[x' = sR \cdot x + t \]

\todo{Verificar link. \\
http://docs.opencv.org/doc/tutorials/imgproc/shapedescriptors/moments/moments.html}