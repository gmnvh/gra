\section{Gesture Components for Natural Interaction with In-Car Devices}
\label{Gesture Components for Natural Interaction with In-Car Devices}

\subsection{Autores}

Martin Zobl, Ralf Nieschulz, Michael Geiger, Manfred Lang, and Gerhard Rigoll

\subsection{Referência}

\hyperref[Gesture Control for use in Automobiles]{Gesture Control for use in Automobiles}

%****************************************************************************************
\section{Gesture Control for use in Automobiles}
\label{Gesture Control for use in Automobiles}

\subsection{Autores}

Suat Akyol, Ulrich Canzler, Klaus Bengler, Wolfgang Hahn

\subsection{Tópicos principais}

\begin{enumerate}
\item Iluminação por LED infravermelho de 950nm
\item Segmentação por "global threshold"
\item Rastreamento de contornos por tamanho, centroide, momentos Hu
\item Medição de velocidade e aceleração
\item Seleção de região por fuzzy
\item filtragem do braço
\item classificador 
\end{enumerate}

%****************************************************************************************
\section{Computer Vision for Interactive Computer Graphics}

No item 3.1 tem uma figura de um mapa de orientação de uma imagem, mas não diz como calcular esse mapa.

Nesse site mostra um exemplo de como calcular o mapa de orientação:
http://answers.opencv.org/question/9493/fingerprint-orientation-map-through-gradient/

Função do MATLAB para plotar os vetores calculadados:
http://www.mathworks.com/help/matlab/ref/quiver.html

Esse artigo talvez seja uma referencia para esse mapa:
http://www.math-info.univ-paris5.fr/~moisan/papers/dequant.pdf

Funções para converter de cartesiano para polar:
http://docs.opencv.org/modules/core/doc/operations_on_arrays.html#polartocart

A idéia então para plotar o mapa de orientação será simplesmente somar nas coordenadas x e y o valor do gradiente normalizado.
Por exemplo, calcular o vetor em cada 10 pixels ou fazer uma media.