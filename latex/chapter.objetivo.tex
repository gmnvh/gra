\chapter{Objetivo}

O objetivo do trabalho é discutir as principais técnicas para reconhecimento de gestos e poses de mão em um ambiente automotivo.
Os algoritmos e metodologias hoje utilizados para de segmentar e extrair características de imagens e vídeos devem ser estudados e verificados se atingem seu propósito em um ambiente automotivo. Onde há uma forte variação de luz e ausência de controle nas características da mão e do braço do motorista (cor de pele, braço com ou sem vestimentos e vestimentos de cores e estampas diferentes). As características extraídas são utilizadas como entrada em um classificador, responsável pelo reconhecer gestos e poses de mão e assim interagir com o veículo. 

Reconhecimento de gestos baseado em visão é um assunto bastante popular e pesquisado. A busca por mecanismos que tornem a interação entre homem e máquina mais intuitiva e natural.
